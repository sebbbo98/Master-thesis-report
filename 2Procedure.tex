\chapter{Description of the procedure}

BBH Consulting AG, in whose cooperation the preparation of the master thesis is to take place, has conducted three studies on infrastructure planning for the next 25 years in one German region each. The regions are located in North Rhine-Westphalia, Lower Saxony and Schleswig-Holstein and roughly correspond to the size of the area covered by the local network operator, which also commissioned the study. Within these studies, the network area is divided into sub-network areas, which mostly represent a city district or a village (view figure \ref{fig:TNG}). The outcome of these studies is the presentation and description of the most cost-effective transformation path of the energy system and its infrastructure, while respecting the climate goals, based on an energy system model developed by BBHC. The model here is based on the python framework 'oemof'. 

\begin{figure}[htbp]
    \centering
    \includegraphics[width=0.65\linewidth]{Bilder/Teilnetzgebiete_Schwerte_ohne_Namen.png}
    \caption{An example of the division of a region into sub-network areas shown with the open source geographic information system (GIS) software QGIS.}
    \label{fig:TNG}
\end{figure}

The aim of the master thesis is to compare and classify these regional studies with regard to the national climate neutrality studies. 
At first, a generalization of the study results shall take place. For this purpose, a cluster analysis of the sub-network areas will be carried out. Subsequently, the individual clusters and its respective transformation path will be described.
The last and largest part of the master thesis will be the integration or comparison with the current state of research in the form of the above-mentioned national studies.