% Farben
\usepackage{color}           % Alles folgende: \color{red} -- Nur ein Abschnitt: \textcolo{blue}{lorem ipsum}

\usepackage{setspace}        % Zeilenabstand ändern: \begin{doublespace} oder singlespace oder onehalfspace
\usepackage{lipsum}          % Lorem Ipsum - random Text erzeugen: \lipsum[100] (Anzahl Wörter)
\usepackage{pgfplots}        % Plots erstellen https://de.overleaf.com/learn/latex/Pgfplots_package
\pgfplotsset{compat=1.13}    % <- needed!!

\definecolor{myblue}{HTML}{309345}

\usepackage{amssymb}         % https://de.wikibooks.org/wiki/LaTeX-Kompendium:_amssymb
\usepackage{amsmath}
\usepackage{graphicx}

\usepackage[utf8]{inputenc}  %
%Sprachpaket: letzteres ist Standardpaket
\usepackage[ngerman, english]{babel}  % Umlaute im Deutschen
\usepackage[T1]{fontenc}
\setcounter{tocdepth}{3}
\setcounter{secnumdepth}{3}
\usepackage{tabularx}        % Tabellen
\usepackage{multirow}        % span bei Tabellen

\usepackage[backend=biber,style=numeric]{biblatex} % Zitieren
\addbibresource{Verzeichnisse/literatur.bib} % Quellenverzeichnis 

\usepackage{enumitem}        % Manipulation von Listen -- https://de.wikibooks.org/wiki/LaTeX-W%C3%B6rterbuch:_enumitem
\usepackage{csquotes}        % Zitate \cite{} https://www.namsu.de/Extra/pakete/Biblatex.html
\usepackage{float}           % zB Bilder direkt im Text (s. https://en.wikibooks.org/wiki/LaTeX/Floats,_Figures_and_Captions)
\usepackage{caption}         % Bildunter- & überschriften
%\usepackage{textcomp}       % Verschiedene Währungssymbole und co
%\usepackage{rotating}       % Rotating objects \begin{sideways} & \begin{rotate}{30} & \begin{turn}{30}
\usepackage{eurosym}
\usepackage{ltablex}         % Kombiniert tabularx und longtable

\usepackage{microtype}       % automatische Verbesserung der pdf-Datei

%%%%%%%%%%%%%%%%%%%%%%%%%%%%%%%%%%%%%%%%%%%%%%%%%%%%%%%%%%%%%%%%%%%%%%%%%%%%%%%%%%%%%%%%%%%%%%%%%%%%%%%%%%%%%%%%%%
\usepackage{lmodern}        %schönere Schrift in pdf Dateien
        
%%%%%%%%%% Seitenlayout ändern %%%%%%%%%%%%%%%
% \usepackage{geometry}
% \geometry{
% % showframe,
%  papersize={210mm,297mm},
%  includeheadfoot,
%  top = 20mm,
%  left = 25mm,
%  bottom= 20mm,
%  right = 25mm,
%  lines = 45,
%  headsep = \baselineskip,
%  footskip = \dimexpr2\baselineskip+3mm\relax,
% }

\usepackage{scrlayer-scrpage}   % 

\newcommand{\HRule}{\rule{\linewidth}{0.2mm}}

\KOMAoptions{ 
	listof=totoc,% LoF, LoT ins ToC eintragen 
	listof=entryprefix,% Prefix für Einträge in LoF, LoT etc. verwenden 
	listof=indent% Aufheben von listof=flat 
} 

%%%%%%%%%% Hat irgendwas mit den Verzeichnissen zu tun
\newcaptionname{ngerman}{\listoflofentryname}{Abb.} 
\newcaptionname{ngerman}{\listoflotentryname}{Tab.}

\usepackage{expl3} 
\ExplSyntaxOn 
\clist_map_inline:nn 
{figure,table} 
{\DeclareTOCStyleEntry[ 
	indent=0pt,% kein Einzug 
	dynnumwidth,% automatisches Bestimmen des Platzes für die Nummer 
	numsep=2em% Abstand zwischen Nummer und Eintragungstext 
	]{section}{#1} 
} 
\ExplSyntaxOff 

\AfterTOCHead{\thispagestyle{empty}\pagestyle{empty}} 
\AfterStartingTOC{\cleardoublepage} 