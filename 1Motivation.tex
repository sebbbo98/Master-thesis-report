\chapter{Background and motivation}

The fact that combating climate change has gained importance in politics is demonstrated by the German government, which amended the 'Klimaschutzgesetz' in November 2022 and is now targeting the achievement of climate neutrality by 2045 at the latest \cite{Klimaschutzgesetz2022}. However, at least as important is the implementation of this transformation, and here the law does not specify a particular path. At the moment, it's one of the most important tasks of science and economics to point out possible paths to the specified goal and derive recommendations for action from it for the respective actors in politics, business, science and the economy. 

\section{Current state of research}

For this reason, there are various studies at national level that attempt to map a transformation path for the entire German energy system. A selection of these studies can be found in table \ref{tab:Klimaneutralitaetsstudien}. 

\begin{table}[htbp]
\caption{The five biggest studies which which adress the transformation to a climate neutral germany in 2045.}
\resizebox{\columnwidth}{!}{%
\begin{tabular}{|l|c|c|c|}
\hline
\textbf{Study} & \textbf{Published} & \textbf{Contracting authority} & \textbf{Climate neutrality} \\ \hline
Klimaneutrales Deutschland 2045 \cite{Agora2021} & 06/2021 & \begin{tabular}[c]{@{}c@{}}Stiftung Klimaneutralität, Agora\\ Energiewende, Agora Verkehrswende\end{tabular} & 2045 \\ \hline
Aufbruch Klimaneutralität \cite{dena2021} & 10/2021 & dena & 2045 \\ \hline
\begin{tabular}[c]{@{}l@{}}Langfristszenarien für die Transformation \\ des Energiesystem in Deutschland 3 \cite{BMWK2021}\end{tabular} & 10/2022 & BMWK & 2045 \\ \hline
Deutschland auf dem Weg zur Klimaneutralität \cite{Ariadne2021} & 10/2021 & Kopernikus-Projekt Ariadne & 2045 \\ \hline
\begin{tabular}[c]{@{}l@{}}Strategien für eine treibhausgasneutrale \\ Energieversorgung bis zum Jahr 2045\end{tabular} \cite{Jülich2022} & 11/2021 & Forschungszentrum Jülich (IEK-3)  & 2045 \\ \hline
\end{tabular}%
}
\label{tab:Klimaneutralitaetsstudien}
\end{table}

These studies describe the transformation across all sectors and all technologies and also quantify it for Germany as a whole.
In the literature, however, there are few studies so far that deal specifically with the regional transformation of energy systems. For example Limpens et al. \cite{Limpens2019} describe a regional energy system model applied in Switzerland, though it is focused on describing the model. No specific results of the model or any derived recommended actions for the transformation are discussed. Schmid et al. \cite{Schmid2016} describe the exemplary transformation of the energy system in Germany, but the description does not take place from a technological perspective, nor is it quantified more precisely. This is where the topic of this work picks up on. 

\section{Research question}